\documentclass[10pt, notitlepage, a4paper]{article}

\usepackage[pdftex]{graphicx, color}
\usepackage[pdftex,colorlinks,hyperfigures,pagebackref,hyperindex,linkcolor=black,anchorcolor=black,citecolor=black,pagecolor=black]{hyperref}
\usepackage{verbatim}

\begin{document}

\title{NRS DataLogger Documentation}
\author{$Revision: 1.1 $}
\maketitle

\section{DataLogger}
The NRS {\it Datalogger} is a Java-based NRS component for real-time logging of
data from NRS components. By linking data sources to the logger, it is possible
capture and write out data of various NRS basic types. This is accomplished by
components sending data messages to specific nodes, which write the data out to
a file. 

Note that this component has no front-end user interface. Control of the
component setup is done via the NRS.gui using nodes and variables.

The {\it DataLogger} component has a simple structure, with one root node and five types of value nodes. The root node, called {\it DataLoggerNode}, acts as a
container for the typed-nodes. The root node is also used to setup initial
parameters, as node attributes, that apply to all value nodes, i.e. directory to
store log files in, initial time and the timestep (more on these later). These
initial values are propagated down to any child nodes (typed-nodes), and can be
changed using the variables: {\it Directory Name}, {\it Time} and {\it
  TimeStep}. More detail on these variables in table \ref{root_nodes_vars}.

\begin{table}[!ht]
\begin{center}
\label{root_node_vars}
\caption{DataLoggerNode variable}
\begin{tabular}{|c|l|l|l|}
\hline Variable & Type & Interfaces & Comments\\ \hline
\hline Directory Name & Filename & input/output & must be absolute path\\
\hline Time & float & input/output & current time\\
\hline TimeStep & float & input/output & timestep (e.g. 1 ms)\\
\hline Counter & void & input & increment time by timestep\\
\hline Reset & void & output & \\
\hline
\end{tabular}
\end{center}
\end{table}

To log data, an appropriately typed value node is required. The currently
available typed-nodes are shown in table \ref{logger_nodes}. For example, to
log integers, an {\it IntegerNode} would be required. In fact, the interface
to all typed-nodes is identical, i.e. same attributes and variables. All that is
required is to specify a filename as an attribute to the node. It is also
possible to set an attribute to specify whether, if a file already exists, the
log data should be appended to file or just overwrite it (default). It is 
possible to change these values using the same named variables: {\it Filename}
and {\it AppendToFile}. To send logging data it is necessary to link the source
variable (i.e. the one producing the data) to the {\it Value} variable of an
appropriate typed-node. It is possible to connect various sources to one node,
but this is not very useful, as there is no distinction between the sources of
data. 

\begin{table}[!ht]
\begin{center}
\label{logger_nodes}
\caption{DataLogger typed-nodes}
\begin{tabular}{|c|l|l|}
\hline Node & Type\\ \hline
\hline BooleanNode & boolean\\
\hline IntegerNode & integer\\
\hline FloatNode & float\\
\hline StringNode & string\\
\hline VoidNode & void\\
\hline
\end{tabular}
\end{center}
\end{table}

When a node receives a data value, it simply writes this value and the current
time to a file. The structure of the file is two columns, one with time and one
with the data value. The exception to this rule is for {\it VoidNode}s, where
there is only one column: time, as there is no logging data.

\subsection{What time is it?}
The {\it DataLogger} component offers two mechanisms for maintaining time. One
involves an external time source and the second involves using relative time,
triggered by an external source. In more detail, for the first system it is
possible to connect up the time source of another NRS component, e.g. the NSim
(see below for comments on this), to the {\it Time} variable of the {\it
  DataLoggerNode} directly. This means that as the {\it NSim} runs and updates
its time, the {\it DataLogger} will keep the same time. The second mechanism
allows for the maintenance of relative time. This is triggered by the receipt of
void messages at the {\it Counter} variable. On receipt of these messages, the
node updates its internal value of time to be the current time value plus one
timestep value. This simulates time, and is useful for components which don't
maintain real-time information, e.g. a robot.

\subsection{Recommendations for use}
There are issues with using the {\it NSim}'s {\it SimulationTime} variable for
maintaining time on the {\it DataLogger}. When the granularity is set very fine,
e.g. 1 ms, the volume of messages sent from the NSim is so large that it is not
possible for the {\it DataLogger} to cope. It thus falls behind and messages get
clogged up in a queue. This is not reliable. The recommendation is to use a
smaller granularity, or even better would be to create a variable which is
attached to a components {\it MainLoop} variable, that sends void updates to the
{\it Counter} variable of the {\it DataLoggerNode}. 

\end{document}
