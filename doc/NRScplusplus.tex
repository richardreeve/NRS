%% $Header: /disk/cvs/cricketlab/software/new/doc/NRScplusplus.tex,v 1.2 2005/10/25 15:51:38 richardr Exp $
%% Early on we need a \documentclass or a \documentclass
%% If we want to use the new features of LaTeX2e, the document
%% should begin with a \documentclass rather than \documentclass
%%
%%\documentclass[a4paper,11pt,draft]{article}
\documentclass[pdftex,a4paper]{article}

%%
%% Determine is we are running pdflatex or just latex
%%
\newif\ifpdf
\ifx\pdfoutput\undefined
    \pdffalse           % we are not running PDFLaTeX
\else
    \pdfoutput=1        % we are running PDFLaTeX
    \pdftrue
\fi

%% Uncomment out these lines for Latex to check syntax only (faster)
%% \usepackage{syntonly}

\usepackage{natbib}
\usepackage{graphicx}
\usepackage{times}   % better for displaying on a screen
\usepackage[pdftex,colorlinks,hyperfigures,pagebackref,hyperindex,linkcolor=black,anchorcolor=black,citecolor=black,pagecolor=black]{hyperref}

\newcommand{\ie}{{\em i.e.\ }}
\newcommand{\eg}{{\em e.g.\ }}
\newcommand{\egg}{{\em e.g.}}
\newcommand{\etc}{{\it etc.}}
\newcommand{\et}{{\it etc}}
\newcommand{\via}{{\it via }}
\newcommand{\vs}{{\it vs. }}
\newcommand{\viz}{{\it viz. }}
\newcommand{\adhoc}{{\it ad hoc}}
\newcommand{\etal}{{\it et al.\ }}
\newcommand{\ibid}{({\it ibid.})}

\newcommand{\NRS}{{\it NRS}}
\newcommand{\NRSgui}{{\it NRS.gui}}
\newcommand{\NRSnsim}{{\it NRS.nsim}}

\DeclareRobustCommand{\trademark}{\ensuremath{^{\mathrm{TM}}}}

\newcommand{\XML}[2][]{{\tt \small $<$#2#1/$>$}}
\newcommand{\XMLfont}[1]{{\tt \small #1}}
\newcommand{\code}[1]{{\tt \small #1}}

\newcommand{\XMLtext}[1]{
  \begin{tt}
    \begin{small}
      \begin{list}{}{
          \setlength{\topsep}{0pt}
          \setlength{\partopsep}{0pt}
          \setlength{\itemsep}{0pt}
          \setlength{\parsep}{0pt}
          \setlength{\leftmargin}{2em}
          \setlength{\rightmargin}{2em}
          \setlength{\labelsep}{0pt}
        }
      \item #1
      \end{list}
    \end{small}
  \end{tt}
}

\newcommand{\XMLfull}[3][]{\XMLtext{$<$#2#1$>$
    #3
  \item $<$/#2$>$}}

\newcommand{\XMLsimple}[2][]{\XMLtext{$<$#2#1/$>$}}

%% \syntaxonly

%% Define the title
\author{$Revision: 1.2 $}

\title{NRS 2.0 C++ Design Document}

\hypersetup{pdfauthor = {Richard Reeve},
  pdftitle = {NRS 2.0 C++ Design Document},
  pdfkeywords = {C++, NRS}
}

% Change paragraph spacing to word-like documents
%\setlength{\parindent}{0pt}
%\setlength{\parskip}{2ex plus 0.5ex minus 0.2ex}


%%\includeonly{model}

%% ---------------------------------------------------------
%% -------------------- END OF PREAMBLE --------------------
%% ---------------------------------------------------------

\begin{document}

\date{}
%% generates the title
\maketitle

%% insert the table of contents
\tableofcontents

\pagebreak

\section{Introduction}

This document summarises how to add a new component to the NRS C++
framework.

Each component is compiled in C++ as a shared library which is then
loaded at runtime to add its capabilities to the core C++
NRS.component program. This means that a C++ component can
theoretically be maintained and compiled completely separately from
the main code. In practice this is not done as the shared libraries
will of necessity reference the core headers.

Building a shared library inside the main NRS tree to act as an NRS
component happens in four stages:

\begin{enumerate}

\item Setting up a new directory and configuring the Makefile.am file
  inside the directory and the configure.ac in the root directory so
  that it will be built.

\item Creating a simple base library file in the directory which
  reports on the insertion and removal of the plugin. {\em At this
  stage the component can be built and inserted and removed (though it
  will do nothing).}

\item Creating the root node for the component, and any new variable
  types. This node is the only compulsory node for a component, and in
  the simplest case can often be copied directly from another
  component. In other cases, where the component is merely a wrapper
  for another (non-NRS) program, this may be the only node in the
  component, but will need to be created itself. {\em At this stage
  loading the component will register with the GUI, and the root node
  can be created and destroyed.}

\item Creating the other nodes and variables for the component. This
  is the main job for most components, and can involve the design and
  creation multiple node and variable types. It also requires that, in
  a hierarchical system, the parent nodes are set so that they can
  contain their children types. {\em Once all the necessary node and
  variable types have been created, the component is finished.}

\end{enumerate}

\section{Directory Creation}

This is a relatively easy task, and much can be copied from a sample
component directory.

First create a directory in the name of the component, and copy the
{\tt Makefile.am} and the {\tt .cvsignore} files from the {\tt
sample/} directory, and edit the file to replace all references to
{\em sample} with the name of the component. Then add these files to
the cvs repository using {\tt cvs add {\em sample}} and then {\tt cvs
add {\em sample}/Makefile.am} and {\tt cvs add {\em
sample}/.cvsignore}; this is then completed by committing the files to
the cvs repository using {\tt cvs commit {\em sample}}.

Next an alias must be set up for the new component. In the {\tt
component/Makefile.am} file, the string ``INSERT:'' can be found,
along with exactly what to add in there.

Finally the root level {\tt configure.ac} file must be edited to
include the new component directory in the build. In two places in the
file, again the string ``INSERT:'' can be found, along with
instructions about what exactly to add in there.

\section{Base Library Creation}

Again, a very simple task. This is required to allow the main
NRS.component executable to recognise the library as an acceptable
plugin. Simple copy the {\tt sample/SampleLibrary.cc} file into the
new directory with an appropriate name (as used in the {\tt
Makefile.am}), and edit the file to replace references to {\em sample}
with the component name. At this point a namespace also has to be
chosen for the component. The file can now be added to cvs.

At this point, it should be possible to build the software to test
that everything has been done correctly so far. First run {\tt
autoreconf} in the root of the source directory, and then run {\tt
configure} in the object directory as detailed in the build
instructions and build the software. A stub plugin will be created and
the newly aliased executable will now run, though obviously it will do
nothing except report its loading and unloading.

\section{Root Node Creation}

In most cases, one of the default root nodes can just be copied, which
is another simple task. This is possible unless the root node is
required to serve some specific function, such as to start up a
external program, where considerable customisation may need to be
done. This section of the documentation covers the simple case;
creating a customised root node can be seen as just using advice in
the next section for creating other nodes to customise one of the
existing sample root nodes.

There are a few different component types, which require solutions of
different complexities. However, all of them share several files in
common. First copy {\tt Sample.cc}, {\tt Sample.hh} and {\tt
SampleManager.cc} across to the new directory from the {\tt sample/}
directory, rename them appropriately, and rename the uses of {\em
sample} inside the files. Note that the class names should match the
filenames, and the namespace should be the same as that used in the
library file above. These filenames can now be added to the
Makefile.am at the ...\_SOURCES line, and then the files can be added
to cvs.

Then a sample factory file has to be chosen.

\begin{itemize}

\item In the simplest and probably rarest case the component will
contain elements (nodes) which do not need to connect to each other,
and so timing message sending and updating is unimportant, and it does
not matter what order they happen in. If in doubt do not use this,
however it may be used as a starting point when an external program is
being incorporated into NRS. The file in this case is {\tt
SampleFactory0.cc} and should be copied into the target directory with
an appropriate name ({\tt xxxFactory.cc}). Child nodes of root nodes
of this type (if there are any) must have a MainLoop variable which
triggers all actions in the node including updating of variable state
and the sending of any outbound messages.

\item The next simplest case is for systems where the component
contains real NRS nodes, which are potentially connected to each
other, and so we need to make sure that all nodes update their state
at the same time and then send out messages together, otherwise the
system may act differently depending on which order the nodes are
activated in. In this case use {\tt SampleFactory1.cc} as above. Child
nodes of root nodes of this type must have Updater and MessageSender
variables, which must trigger an update of the internal state of the
variables in the node and intra-node communication of that state in
the first case, and external communication (to other nodes inside the
component, and other components) in the second.

\item The final case is for systems which need to explicitly model
time, such as simulators. Such nodes always (in our experience) need
to correctly synchronise message passing, though your experience may
potentially be different. In any event, this root node knows
explicitly about time, so that it can pass on the information to all
of its child nodes, but it requires that a sub-node exists to actually
count the time down. This is done so that the same component can be
made to run locked to real time, or as fast as possible, or whatever,
by just changing the Timer sub-node being used. In this case use the
{\tt SampleFactory2.cc} file as above. All child nodes except the
Timer node must have Updater and MessageSender variables as in the
second case, and may in addition have Timestep and SimTime variables
if they want to keep track of the time of the simulation. The Timer
node must have a MainLoop variable which triggers the timer
incrementing, and the Updater and MessageSender variables then being
activated.

\end{itemize}

\section{General Node Creation}

It would be impossible to give details on how to create an arbitrary
node to carry out some unknown function. There are a few tips that can
be given however. First of all, you have to think in terms of
variables and messages --- any piece of information or any event that
needs to be transferred from one place to another or which needs to be
recorded at some point must be stored in a variable, and transmitted
in a message.  It will usually be sent to another variable of the same
type; the only time this rule can be broken is for messages being
passed inside a node, where sometimes the target can be a variable of
a different type.

For instance, the piece of information may be the current timestep of
a simulation, stored in a variable called Timestep. Many other
variables may need to know about this, for instance the membrane
potential of a neuron (another variable) will need to know about it to
know how far into the future to integrate to calculate its new
state. This is not the same variable type as a timestep, so there must
be a local copy of the timestep inside the neuron, so that a local
connection can be made. This means straight away that every element of
a simulation must have a Timestep variable.

Another example in a different vein is the Updater and MessageSender
variables. These transmit the message to variables inside all nodes
that they should update their internal state and send out
messages. Again, because the variables that are carrying out these
actions are not of type Updater and MessageSender (they can't be both
anyway), a local copy of these variables must exist inside each node.

Variables of these types which carry information from one part of the
program to another, or which just hold some fixed value such as the
capacitance of a neural membrane have to be declared in the Factory
for a node (as you can see in the root node factory's setVariables
method, and we will talk about more below), but creating them is
handled by the main program, and the only other detail you may have to
think about is whether you want to be able to set any default value
for them in the GUI; apart from that everything is handled
automatically. These variables are also immune to the normal
Updater/MessageSender cycle which we have mentioned before, as they
only update their state when triggered externally. Several of them
will also automatically connect to the same variable on their parent
node, so that when the Updater variable in a parent is triggered, the
child will also be triggered. This is carried out by the GUI, and
simplifies life slightly, but is not compulsory.

The variables which have to be written by hand are those which update
their own internal state, such as the angle of a joint, or the
membrane potential of a neuron. They have to be explicitly connected
in the factory to all the variables inside a node which they depend
on, particularly the Updater and MessageSender variables, but also any
others whose state they need to know.

In the case of the membrane potential of a neuron for instance, this
is a great number, including the timestep of the simulation, the
membrane capacitance, the base membrane potential, the base
conductance of the membrane and inputs from the synapses.

Connections like this, from one variable to another of a potentially
different type, are done through MessageFunctionObjects --- this is
the only way of connecting variables of different types together, and
can only be done by the factory. This is why variables of different
types can only be connected together if they are inside the same
node. These connections are also permanent and unbreakable. All other
connections can be made or broken through the GUI with the sole
exception of the MainLoop variable of the root node which is
automatically connected to the main component framework as soon as the
node is constructed, and cannot be disconnected without destroying the
node.

Again, some of this can be handled by the factory's default code, so
that if you want a variable to be told to update when an Updater
variable is triggered, then you can connect them together using a
UpdaterMFO, and likewise you can use a MessageSenderMFO to trigger
messages to be sent, and MainLoopMFO to do both sequentially for nodes
which just have a MainLoop input such as the root node in
SampleFactory1.cc. However, anything else has to be handled by custom
code written by the user. Some examples of this are to be found in
existing component plugin directories such as NSim.

In essence, the system has been written so that as much as possible
is done in the core code, but anything different --- \ie what makes
your component or node actually do something interesting ---
inevitably has to be written by hand.

The best way of seeing how to do something is probably to look for an
example in existing code. Examples particularly include the NSim
Timer, PoissonNeuron, TimeCheck, Evaluator amongst others and the
completed parts of the SpikingNeuron code, though this is emphatically
not complete at the time of writing. You can check in the {\tt
Makefile.am} to see which files are finished, and in the {\tt
xxxFactory.cc} files to see which nodes use which variables.

\end{document}
