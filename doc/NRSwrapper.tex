\documentclass[10pt, notitlepage, a4paper]{article}

\usepackage[pdftex]{graphicx, color}
\usepackage[pdftex,colorlinks,hyperfigures,pagebackref,hyperindex,linkcolor=black,anchorcolor=black,citecolor=black,pagecolor=black]{hyperref}
\usepackage{verbatim}

\begin{document}

\title{NRS Wrapper Documentation}
\author{$Revision: 1.1 $}
\maketitle

\section{Wrapper}
NRS.wrapper is a Java-based NRS component for executing non-NRS
applications via the command-line. Programs are executed using nodes and
variables and run as a separate process in the Java application. Using
NRS attributes it is possible to run these applications with a set of
command-line arguments. When executing, each application's output stream
is monitored and printed out to console of the NRS.wrapper
application. Note that this component has no front-end user
interface. All control of running applications is done through NRS variables.

This component offers one root node, called {\it WrapperNode}, which
acts as a container for {\it ExecuteNode}s. It is the {\it ExecuteNode}s
which are used to run applications. In fact, one {\it ExecuteNode} represents
one application to be run. However, each application can be run multiple
times. Also, there is no limit on the number of {\it ExecuteNode}s which
a {\it WrapperNode} can contain.

On creation of an {\it ExecutionNode}, the application name and any
program arguments must be set as attributes of the node. To change
the command, it is only necessary to edit the attributes of the node.

Each {\it ExecuteNode} has three variables: {\it Start, Stop} and {\it
  Running}. These are summarised in table \ref{execute_node_vars}
and explained further below.

\begin{table}[!ht]
\begin{center}
\label{execute_node_vars}
\caption{ExecuteNode's variables}
\begin{tabular}{|c|l|l|}
\hline Variable & Type & Interfaces\\ \hline
\hline Start & void & input only\\
\hline Running & boolean & input/output\\
\hline Stop & void & input/output\\
\hline
\end{tabular}
\end{center}
\end{table}

The {\it Start} variable, as it's name suggests, is used to start a
program executing. The application then runs until completion or until
it gets interrupted. Sending multiple messages to this variable while a
program is executing has no effect. To restart a process it is first
necessary to stop it and then start it up again. To run multiple
processes simultaneously using the same command it is necessary to
create multiple {\it ExecuteNode}s. 

The {\it Running} variable has two uses. The first is to indicate the
status of a process. When a process is started, the {\it Running}
variable is in a {\tt true} state, then when the process finishes the
variable is in a {\tt false} state. The second use is to set the state
of a process. That is, by setting the state of the variable, the state
of the running process is set accordingly. If the variable is set to a
true state and no process is running, a process is started. If one is
currently running, it is immediately interrupted. In any other case, no
steps are taken.

Finally, the {\it Stop} variable, which is similar to the {\it Running}
variable, can be used to stop a running process. It can also be used to
indicate that a process has completed.

\subsection{Implementation issues}
When an application begins to execute two Java Threads are started. One is
used to monitor the current status of the process, and the second one is used to
monitor the output stream of the process. The first thread blocks on a call to
the {\tt waitFor()} method of a Java {\tt Process} and completes when the
process finishes. The second thread works by looping over the output stream of
the process, displaying any characters output to this stream.

It would be easy to monitor the error stream of a running process as well. This
could be done by adding another monitoring thread. However, currently, this
functionality is not seen as required.

\end{document}
